\documentclass[pdftex,a4paper,DIV15]{scrartcl}

\usepackage{cmap}
\usepackage[utf8]{inputenc}

\usepackage[francais]{babel}
\usepackage{url} 
\usepackage{array}
\usepackage{amsmath}
\usepackage{booktabs}
\usepackage{paralist}
\usepackage{hyperref}

\hyphenation{Map-Reduce}

\title{
PageRank\footnote{nous remercions Pierre Senellart (\texttt{pierre.senellart@telecom-paristech.fr}) pour avoir préparé ce projet. } }
\author{Oana Balalau (TA), Luis Gallaraga (TA), Mauro Sozio \\
(\texttt{firstname.lastname@telecom-paristech.fr})}


\begin{document}

\maketitle



\section*{Calcul de PageRank dans Simple English
Wikipedia}

PageRank est la technique qu'ont proposé les fondateurs de Google, Brin
et Page, pour associer un score aux pages du Web. L'idée de PageRank est
la suivante: \emph{les pages importantes sur le Web sont les pages étant
pointées par des pages importantes}. Plus généralement, le score de
PageRank est une mesure utile à calculer dans tout graphe orienté, et
peut révéler des informations sur l'importance et la centralité des
n\oe{}uds de ce graphe.

PageRank est défini comme la probabilité qu'un \emph{surfeur aléatoire}
effectuant une marche aléatoire sur le Web en suivant les liens
uniformément au hasard (et, avec une faible
probabilité, effectuant un saut vers une autre page du Web choisie
uniformément au hasard) se retrouve sur une page donnée dans un point
distant du futur (une fois que la \emph{mesure d'équilibre} de la chaîne
de Markov a été atteinte).

Étant donné un graphe orienté de matrice d'adjacence $G$ ($G(i,j)$ vaut
$1$ s'il y a un lien de $i$ vers $j$, $0$ sinon), le PageRank des
n\oe{}uds du graphe peut être calculé de la manière suivante:
\begin{enumerate}
  \item Normaliser $G$ pour que chaque ligne somme à 1.
  \item Soit $u$ le vecteur uniforme de somme $1$, soit $v$ égal à $u$.
  \item Répéter jusqu'à convergence (par exemple différence relative
    inférieure à 1\% entre les versions successives de $v$):
    \begin{itemize}
      \item $v:=(1-d){}^tGv + du$ (avec par exemple $d=\tfrac 1 4$).
    \end{itemize}
\end{enumerate}

Vous trouverez dans l'achive un jeu de données formé du graphe
de la version Simple English (voir \url{http://simple.wikipedia.org/}) de
Wikipedia. Ce jeu de données est formé d'un ensemble de titre d'articles
(labels) et d'un ensemble d'arêtes(edge\_list.txt), décrit par un fichier dont 
chaque ligne est de la forme:
\begin{verbatim}
A B1,C1 B2,C2 ... Bn,Cn
\end{verbatim}
où \verb|A| est l'index d'un article (le fichier des titres d'articles
les donnant dans l'ordre), \verb|B1|, \dots, \verb|Bn| sont des index
d'articles pointés par \verb|A|, et \verb|C1|, \dots, \verb|Cn| sont le
nombre de liens de
\verb|A| à l'article en question.

En utilisant MapReduce, le but est de calculer
le PageRank de l'ensemble des n\oe{}uds du jeu de données, et de trier le
résultat par PageRank décroissant.

En particulier vous devrez:
\begin{itemize}
  \item charger le jeu de données dans HDFS, sous un format lisible par
    Hadoop (le format \texttt{SequenceFile} est recommandé, il peut
    être produit avec \texttt{hadoop.writetb}, cf.
    \url{http://hadoopy.readthedocs.org/en/latest/tutorial.html} : complétez le fichier LoadIntoHDFS.py.
  \item écrire la multiplication matricielle sous la forme d'un job
    MapReduce: complétez le fichier PageRank.py;
  \item écrire la structure générale du programme faisant appel à ces
    jobs jusque convergence: complétez le fichier PageRankDriver.py;
  \item trier et interpréter le résultat.
\end{itemize}

Quel est l'article le plus important dans Simple English Wikipedia?


\end{document}

